\documentclass[a4paper,12pt]{scrartcl}
\usepackage{xcolor}
\usepackage[UKenglish]{isodate}%http://ctan.org/pkg/isodate
\usepackage[a4paper,left=2cm, right=2cm,top=2.7cm, bottom=3cm]{geometry}
\usepackage{changepage}
\usepackage{fancyhdr}
\usepackage[export]{adjustbox}
\usepackage{sectsty}
\usepackage[titles]{tocloft}
\usepackage{hyperref}
\usepackage{graphicx}
\hypersetup{
    colorlinks,
    citecolor=black,
    filecolor=black,
    linkcolor=black,
    urlcolor=black
}

\hyphenchar\font=-1
\setlength{\cftbeforesecskip}{4pt}

\newenvironment{subs}
{\adjustwidth{3em}{0pt}}
{\endadjustwidth}
\renewcommand{\cftsecleader}{\cftdotfill{\cftdotsep}}
\renewcommand{\familydefault}{\sfdefault}

\pagestyle{fancy}
\fancyhf{}
\renewcommand{\headrulewidth}{0pt}
\lhead{\today}
\chead{System Design Document}
\rhead{\ProjectName}
\lfoot{\includegraphics[scale=0.147,valign=c]{EIST.png}}
\cfoot{\thepage}
\rfoot{\includegraphics[scale=0.4,valign=c]{TUM.png}}

\newcommand{\ProjectName}{TUM Social}

\begin{document}
    \section*{Purpose}
    The system design is documented in the System Design Document (SDD). It describes additional design goals set by the software architect, the subsystem decomposition (with UML class diagrams), hardware/software mapping (with UML deployment diagrams), data management, access control, control flow mechanisms, and boundary conditions. The SDD serves as the binding reference document when architecture-level decisions need to be revisited.

    \section*{Audience}
    The audience for the SDD includes the system architect and the object designers as well as the project manager.
    \nopagebreak

    \renewcommand{\contentsname}{Table of Contents}
    \tableofcontents
    \section*{Document History}

    \begin{tabular}{
        |p{\dimexpr.08\linewidth-2\tabcolsep-1.3333\arrayrulewidth}% column 1
        |p{\dimexpr.17\linewidth-2\tabcolsep-1.3333\arrayrulewidth}% column 2
        |p{\dimexpr.20\linewidth-2\tabcolsep-1.3333\arrayrulewidth}
        |p{\dimexpr.56\linewidth-2\tabcolsep-1.3333\arrayrulewidth}|% column 3
    }
        \hline
        Rev. & Author & Date            & Changes          \\
        \hline
        1    & name   & 31st April 2022 & Sample changes 1 \\
        \hline
        2    & name   & 15th May 2022   & Sample changes 1 \\
        \hline
        3    & name   & 29th May 2022   & Sample changes 1 \\
        \hline
        4    & name   & 12th June 2022  & Sample changes 1 \\
        \hline
        5    & name   & 26th June 2022  & Sample changes 1 \\
        \hline
    \end{tabular}
    \newpage
    \sectionfont{\color[HTML]{355a8a}}  % sets colour of chapters
    \subsectionfont{\color[HTML]{4e81bc}}


    \section{Introduction}
    The purpose of the Introduction is to provide a brief overview of the software architecture. It also provides references to other documents.
    \begin{subs}
        \subsection{Overview}

        \subsection{Definitions, acronyms, and abbreviations}

        \subsection{References}
    \end{subs}


    \section{Design Goals}
    This section describes the design goals and their prioritization (e.g. usability over extensibility). These are additional nonfunctional requirements that are of interest to the developers. Any trade-offs between design goals (e.g., usability vs. functionality, build vs. buy, memory space vs. response time), and the rationale behind the specific solution should be described in this section. Also the rationale of all other decisions must be consistent with described design goals.


    \section{Subsystem Decomposition}
    This section describes the decomposition of the system into subsystems and the services provided by each subsystem. The services are the seed for the APIs detailed in the Object Design Document.


    \section{Hardware/Software Mapping}
    This section describes how the subsystems are mapped onto existing hardware and software components. A UML deployment diagram accompanies the description. The existing components are often off-the-shelf components. If the components are distributed on different nodes, the network infrastructure and the protocols are also described.


    \section{Persistent Data Management}
    This section describes how the entity objects are mapped to persistent storage.
    It contains a rationale of the selected storage scheme, file system or database, a description of the selected database and database administration issues.


    \section{Access Control and Security}
    This section describes the access control and security issues based on the nonfunctional requirements in the requirements analysis document. It also describes the implementation of the access matrix based on capabilities or access control lists, the selection of authentication mechanisms and the use of encryption algorithms.


    \section{Global Software Control}
    This section describes the control flow of the system, in particular, whether a monolithic, event-driven control flow or concurrent processes have been selected, how requests are initiated and specific synchronisation issues.


    \section{Boundary Conditions}
    This section describes the use cases how to start up the separate components of the system, how to shut them down, and what to do if a component or the system fails.


\end{document}