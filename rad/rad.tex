\documentclass[a4paper,12pt]{scrartcl}
\usepackage{xcolor}
\usepackage[UKenglish]{isodate}%http://ctan.org/pkg/isodate
\usepackage[a4paper,left=2cm, right=2cm,top=2.7cm, bottom=3cm]{geometry}
\usepackage{changepage}
\usepackage{fancyhdr}
\usepackage[export]{adjustbox}
\usepackage{sectsty}
\usepackage[titles]{tocloft}
\usepackage{hyperref}
\usepackage{graphicx}
\hypersetup{
    colorlinks,
    citecolor=black,
    filecolor=black,
    linkcolor=black,
    urlcolor=black
}

\hyphenchar\font=-1
\setlength{\cftbeforesecskip}{4pt}

\newenvironment{subs}
{\adjustwidth{3em}{0pt}}
{\endadjustwidth}
\renewcommand{\cftsecleader}{\cftdotfill{\cftdotsep}}
\newenvironment{subsubs}
{\adjustwidth{2em}{0pt}}
{\endadjustwidth}
\renewcommand{\cftsecleader}{\cftdotfill{\cftdotsep}}
\renewcommand{\familydefault}{\sfdefault}

\pagestyle{fancy}
\fancyhf{}
\renewcommand{\headrulewidth}{0pt}
\lhead{\today}
\chead{Requirements Analysis Document}
\rhead{\ProjectName}
\lfoot{\includegraphics[scale=0.147,valign=c]{EIST.png}}
\cfoot{\thepage}
\rfoot{\includegraphics[scale=0.4,valign=c]{TUM.png}}


%%%%----TODO: Change your project name----%%%%
\newcommand{\ProjectName}{Name of your project}
%%%%----TODO: Change text from here-------%%%%
\begin{document}
    \section*{Purpose}
    The results of the requirements elicitation and the analysis activities are documented in the Requirements Analysis Document (RAD). This document completely describes the system in terms of functional and nonfunctional requirements and serves as a contractual basis between the client and the developers.

    \section*{Audience}
    The audience for the RAD includes the client, the end users, the project manager, and the developers.
    \nopagebreak

    \renewcommand{\contentsname}{Table of Contents}
    \tableofcontents
    \section*{Document History}

    \begin{tabular}{
        |p{\dimexpr.08\linewidth-2\tabcolsep-1.3333\arrayrulewidth}% column 1
        |p{\dimexpr.17\linewidth-2\tabcolsep-1.3333\arrayrulewidth}% column 2
        |p{\dimexpr.20\linewidth-2\tabcolsep-1.3333\arrayrulewidth}
        |p{\dimexpr.56\linewidth-2\tabcolsep-1.3333\arrayrulewidth}|% column 3
    }
        \hline
        Rev. & Author & Date            & Changes          \\
        \hline
        1    & name   & 31st April 2022 & Sample changes 1 \\
        \hline
        2    & name   & 15th May 2022   & Sample changes 1 \\
        \hline
        3    & name   & 29th May 2022   & Sample changes 1 \\
        \hline
        4    & name   & 12th June 2022  & Sample changes 1 \\
        \hline
        5    & name   & 26th June 2022  & Sample changes 1 \\
        \hline
    \end{tabular}
    \newpage
    \sectionfont{\color[HTML]{355a8a}}
    \subsectionfont{\color[HTML]{4e81bc}}
    \subsubsectionfont{\color[HTML]{6b96c7}}


    \section{Introduction}
    The purpose of the Introduction is to provide a brief overview of the function of the system and the reasons for its development, its scope, and references to the development context. The introduction also includes the objectives and success criteria of the project.
    \begin{subs}
        \subsection{Purpose of the System}

        \subsection{Scope of the System}

        \subsection{Objectives and Success Criteria of the Project}

        \subsection{Definitions, acronyms, and abbreviations}

        \subsection{References}

        \subsection{Overview}
    \end{subs}


    \section{Current System}
    This section describes the current state of affairs. If the new system will replace an existing system, this section describes the functionality and the problems of the current system


    \section{Proposed System}
    The third section documents the requirements elicitation and the analysis model of the new system.
    \nopagebreak
    \begin{subs}
        \subsection{Overview}
        The overview presents a functional overview of the system.

        \subsection{Functional Requirements}
        Functional requirements describe the high-level functionality of the system. This section list all functional requirements and additionally presents the dependencies between them.

        \subsection{Non-Functional Requirements}
        Nonfunctional requirements describe user-level requirements that are not directly related to functionality. This includes usability, reliability, performance, supportability, implementation, interface, operational, packaging, and legal requirements. The section list all these non-functional requirements and additionally presents the dependencies between them.
        \begin{subsubs}
            \subsubsection{Usability}

            \subsubsection{Reliability}

            \subsubsection{Performance}

            \subsubsection{Supportable}

            \subsubsection{Implementation Requirements}

            \subsubsection{Interface Requirements}

            \subsubsection{Packaging Requirements}

            \subsubsection{Legal Requirements}
        \end{subsubs}

        \subsection{System Models}
        The System models include scenarios, use cases, object model, and dynamic models for the system. This section should contain the complete functional specification, including mock-ups, paper-based prototypes or storyboards illustrating the user interface of the system and navigational paths representing the sequence of screens.
        \begin{subsubs}
            \subsubsection{Scenarios}

            \subsubsection{Use Case Model}

            \subsubsection{Object Model}

            \subsubsection{Dynamic Model}

            \subsubsection{User Interface}
        \end{subsubs}
    \end{subs}


    \section{Glossary}
    A glossary of important terms used in the project and in the system model ensures consistency in the specification and a common understanding of terms used by the client.

\end{document}